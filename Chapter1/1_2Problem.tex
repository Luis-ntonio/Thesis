\label{section:problema}
Despite the remarkable advancements in Large Language Models (LLMs) across various sectors, including healthcare \cite{mumtaz2024llmshealthcare,10.1093/bioinformatics/btz682}, finance\cite{zhao2024revolutionizing}, and e-commerce, these models often encounter challenges when tasked with domain-specific applications. One significant issue within the e-commerce sector is the effective interaction with detailed product information due to the lack of high-quality, focused datasets \cite{macková2023promap}. Excluding comprehensive datasets like those from Amazon or Wikipedia, this deficiency impacts the ability of LLMs to accurately extract and normalize product attribute values. This limitation results in suboptimal product reviews and recommendations, adversely affecting user experience and decision-making processes.
\\\\
Moreover, the diverse structure and content of product data on e-commerce platforms present additional challenges \cite{liu2019roberta}. Product data can appear in various formats such as JSON, CSV, TSV, and others, complicating the task of LLMs to process and generate structured data effectively \cite{rfc8259}. The JSON format, in particular, is highly valued for its readability and ease of integration with contemporary web technologies, yet leveraging this format for fine-tuning LLMs to enhance their performance remains a critical area of need \cite{ling2024domain}.
\\\\
There is a pressing necessity to create and utilize datasets that cater specifically to the structure and nuances of product data in JSON format. By addressing this gap, the performance of LLMs in accurately processing and generating structured data can be significantly improved. This enhancement is crucial for the generation of more accurate and contextually relevant product reviews, ultimately leading to improved customer satisfaction and engagement on e-commerce platforms \cite{vaswani2023attention}.