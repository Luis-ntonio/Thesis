\label{section:problema}
LLMs have shown impressive abilities in industries like healthcare \citep{He2023ReviewOS}, finance \citep{Varshney_2024}, and e-commerce \citep{peng2024ecellmgeneralizinglargelanguage}, handling all sorts of tasks. But their performance across different domains often suffers because there just aren't enough datasets, especially in e-commerce. Some of the biggest improvements in LLM performance have come from tabular datasets like WikiTable \citep{chen2021wikitabletlargescaledatatotextdataset} and QTSumm \citep{zhao2023qtsummqueryfocusedsummarizationtabular}, which help models do better on tasks like summarization. Even so, e-commerce still lacks high-quality datasets that capture the key details needed for fine-tuning models for these kinds of tasks \citep{macková2023promap}.
\\

E-commerce platforms usually present product data in formats like JSON, CSV, or TSV. While these formats are common, JSON in particular can make it tricky to fine-tune LLMs \citep{gao2024jsontuning}. This makes it harder for models to generate accurate and contextually relevant reviews, which in turn makes it more difficult for users to understand the information and make informed decisions.
\\

On top of that, the absence of specialized datasets means e-commerce platforms struggle to provide users with reliable and consistent information. Bad or incomplete reviews lead to poor customer experiences, higher return rates, and inefficiencies in operations.