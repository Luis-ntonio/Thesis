\label{section:problema}
Across a variety of industries, including healthcare \cite{He2023ReviewOS}, finance \cite{Varshney_2024}, and e-commerce, large language models (LLMs) have shown transformative potential. Notwithstanding their adaptability, these models face considerable difficulties in domain-specific applications, especially in e-commerce. Lack of high-quality, targeted datasets designed to capture the fine details of product features and user interactions is one urgent problem \cite{macková2023promap}. Existing datasets, like WikiTableT \cite{chen2021wikitabletlargescaledatatotextdataset} and QTSUMM \cite{zhao2023qtsummqueryfocusedsummarizationtabular}, are useful but unable to meet the various needs of e-commerce platforms, like producing attribute-specific text or personalized product reviews.

E-commerce platforms frequently display product data in a variety of formats, including JSON, CSV, and TSV. Although JSON is preferred due of its organized and machine-readable nature, it is still difficult to use it successfully for fine-tuning LLMs \cite{gao2024jsontuning}. This restriction has an impact on LLMs' capacity to produce precise and contextually appropriate product reviews, which are essential for raising consumer happiness and facilitating better decision-making on e-commerce platforms.

Additionally, the lack of customized datasets makes it more difficult for e-commerce platforms to give users reliable and consistent information. Incomplete or misaligned reviews frequently contribute to bad customer experiences and higher return rates, which in turn causes operational inefficiencies. To close this gap and improve textual output quality and platform dependability, it is essential to use a specific dataset such as \textbf{eC-Tab2Text}. This will allow models to manage the subtleties of e-commerce product data.
