The motivation for this project stems from the current limitations faced by large language models (LLMs) in effectively handling domain-specific tasks, particularly within the e-commerce sector \cite{Wang2023Emotional}. According to Macková and Pilát \cite{macková2023promap}, a primary challenge is the absence of high-quality, focused datasets tailored to specific product characteristics, which significantly hampers the ability of LLMs to interact efficiently with detailed product information.
\\\\
Fine-tuning existing models with specialized datasets emerges as a strategic solution to bridge this gap. By enhancing model performance through targeted training, these models can better meet the nuanced needs of specific domains such as e-commerce \cite{Duong2023Analysis}. This approach has shown potential in other sectors and is crucial for improving the accuracy and relevance of generated product reviews.
\\\\
Utilizing JSON-centric methods to fine-tune LLMs can significantly improve their ability to process and generate structured data accurately. This is particularly important for e-commerce platforms where product data's structure and content can vary widely. By focusing on JSON-structured data, the project aims to refine the extraction and normalization of product specifications, leading to more accurate and contextually relevant product reviews.
\\\\
The project aims to address these challenges by developing a comprehensive product-related JSON dataset and fine-tuning models like LLama2-chat, Mistral Instruct, and StructLM. The fine-tuned models are expected to demonstrate significant improvements in metrics such as faithfullness and correctness, thereby enhancing their ability to handle structured product data effectively \cite{Suri2023Do}.