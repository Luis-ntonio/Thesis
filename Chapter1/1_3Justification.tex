Addressing the issues raised by the dearth of specialized datasets for e-commerce applications is what inspired this study. According to \cite{macková2023promap} and \cite{Wang2023Emotional}, the lack of targeted, high-quality datasets makes it difficult for LLMs to work with structured product data. A tactical way to close this gap is through fine-tuning, which enables LLMs to modify their general-purpose skills to meet the unique requirements of e-commerce.

This project aims to improve the creation of attribute-specific product reviews by using the recently released \textbf{eC-Tab2Text} dataset. The dataset is specifically designed for training LLMs like LLama2-chat \cite{touvron2023llama}, StructLM \cite{zhuang2024structlm}, and Mistral \cite{jiang2023mistral} since it captures a variety of product attributes and user intents. By enhancing the model's fidelity, accuracy, and fluency, this fine-tuning method seeks to raise the caliber of generated product reviews and customer interaction.

Furthermore, the need for automation that guarantees constant user happiness is rising as e-commerce platforms become more competitive. In addition to filling existing gaps in attribute-specific review generation, models optimized using \textbf{eC-Tab2Text} lay the groundwork for automated, scalable solutions across a range of sectors. This project is a prime example of how domain-specific datasets can improve AI systems, increasing their relevance and influence in real-world situations.