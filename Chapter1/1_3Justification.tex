the motivation of the study realms in the necessity to data in the domain-specific of product reviews. As highlighted by \citep{macková2023promap} and \citep{Wang2023Emotional}, the shortage of targeted, high-quality datasets makes it challenging for LLMs to effectively handle structured product data. Fine-tuning provides a practical solution by adapting LLMs' general capabilities to meet the specific needs of e-commerce.
\\

The goal of this project is to enhance the generation of attribute-specific product reviews using the newly introduced eC-Tab2Text dataset. Designed specifically for training LLMs like LLama2-chat \citep{touvron2023llama}, StructLM \citep{zhuang2024structlm}, and Mistral \citep{jiang2023mistral}, this dataset captures a wide range of product attributes and user intents. Fine-tuning with this data aims to improve the models' accuracy, fluency, and overall quality of the generated reviews, ultimately leading to better customer engagement.
\\

As e-commerce platforms face increasing competition, the demand for automated solutions that consistently ensure user satisfaction is growing. By addressing current gaps in attribute-specific review generation, models fine-tuned with eC-Tab2Text not only improve review quality but also pave the way for scalable, automated solutions across industries. This project showcases the potential of domain-specific datasets to make AI systems more effective and impactful in real-world applications.