\todo{CS4002}
\todo{CS4002}
\todo[color=blue!40]{CS4003}
\todo[color=green!40]{CS4004}
El reensamblaje de objetos 3D rígidos tiene un rol importante en varios campos como la arqueología, la paleontología y la medicina. En la arqueología y la paleontología, la restauración física de los fragmentos encontrados en los sitios de excavación es una tarea difícil y tediosa, en especial, cuando la colección de fragmentos no está completa, está dañada, o dispersa entre otras colecciones. Además, la manipulación manual de los fragmentos puede poner en peligro su integridad física y valor arqueológico \cite{1}. En la medicina, el reensamblaje se puede usar para la reducción de fracturas óseas \footnote{Es un procedimiento para alinear los fragmentos de un hueso correctamente, por ello, es una técnica clave para la cirugía asistida por ordenador \cite{fn1}.}. En este caso, los fragmentos tienen estructuras porosas y superficies de contacto pequeñas y delgadas \cite{5}. \\

Asimismo, el problema de reensamblaje involucra sub tareas como la extracción de características, la segmentación de nube de puntos, el muestreo, la optimización, entre otros. Por lo general, los métodos para resolver este problema tienen dos etapas: la detección de correspondencias entre las caras fracturadas de los fragmentos y la alineación entre ellas para producir un reensamblaje correcto \cite{2}. \\

En la etapa de detección de correspondencia, el enfoque más común es la extracción de características geométricas para determinar la similitud entre fragmentos, ya sea con métodos basados en puntos, curvas y superficies. Li, Geng y Zhou  \cite{3} muestran que los métodos basados en superficies reducen considerablemente el tiempo de ejecución sin disminuir gravemente el \textit{accuracy}. Wang \textit{et al.} \cite{7} mencionan que los métodos basados en curvas simplifican el problema con fragmentos delgados, pero pierden información significativa en el caso de fragmentos gruesos, por lo cual, su uso se restringe al primer caso. \\

Por otro lado, los métodos de la etapa de alineación se pueden clasificar en métodos basados en optimización, en aprendizaje de características y en aprendizaje de transformaciones \cite{9}. Los métodos basados en optimización tienen bases matemáticas fuertes que pueden asegurar convergencia, pero no son resistentes al ruido o valores atípicos \cite{8}. Los métodos basados en aprendizaje de características como \textit{Deep Closest Point} \cite{11} usan redes neuronales robustas para realizar la búsqueda de correspondencia, y, por consiguiente, la transformación de fragmentos. Los métodos basados en aprendizaje de transformaciones generan directamente el objeto ensamblado como producto de la unión de dos fragmentos \cite{10}. Además, se debe resaltar que la mayor debilidad de los métodos basados en aprendizaje es la necesidad de grandes cantidades de datos de entrenamiento. \\

A pesar de que existe investigación considerable en los problemas de correspondencia \cite{12} y alineación \cite{9} de formas 3D basadas en \textit{deep learning} con resultados destacables, su aplicación en el proceso de reensamblaje aún es limitada. Por ello, el propósito de esta tesis es proponer un método basado en \textit{deep learning} para resolver el problema del reensamblaje.
