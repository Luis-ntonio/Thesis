\todo{CS4002}       
\todo[color=blue!40]{CS4003}
\todo[color=green!40]{CS4004}

Para resolver el problema de ensamblaje con fragmentos reales, es necesario procesar modelos con miles de vértices y caras triangulares. Por ello, uno de los primeros problemas que encontramos es el tiempo de ensamblado. Este proceso puede durar horas debido a que la mayoría de los métodos actuales usan descriptores geométricos y procesos iterativos. \\

Por otro lado, los fragmentos de objetos 3D en el campo de la arqueología, paleontología y medicina presentan características variadas como alta curvatura en la superficie no fracturada o pigmentación en diversas partes del objeto, lo cual representa otro problema importante dentro del reensamblaje. Algunos ejemplos de objetos con estas características son la venus de milo, los guerreros de terracota, y los huesos de animales y seres humanos.  \\

Por tales motivos, notamos que resolver el problema de reensamblaje es una tarea difícil y compleja. Por ello, es necesario explorar nuevas técnicas que nos permitan formular un método eficiente y robusto para resolver este problema.