\todo{CS4002}
\todo[color=blue!40]{CS4003}
\todo[color=green!40]{CS4004}

El presente trabajo propone un \textit{framework} de dos etapas (correspondencia y alineación) para resolver el problema de reensamblaje de objetos 3D rígidos. El avance del semestre 2022 - I está compuesto por la arquitectura de cada etapa en el \textit{framework}. Además, se realizarón experimentos con CNNs y el \textit{score} de similitud, donde se obtuvieron resultados interesantes que se esperan extrapolar con \textit{pointnets} para trabajar sobre nubes de puntos.

\begin{comment}
Se deben escribir de manera condensada los logros obtenidos en este proyecto de tesis y las ideas que no han podido desarrollarse en su trabajo. Las conclusiones deben guardar relación tanto con los objetivos específicos como con el objetivo general. %Además, se puede escribir un pequeño párrafo recordando el problema y los objetivos del proyecto de tesis.
\end{comment}
