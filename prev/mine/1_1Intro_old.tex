\todo{CS4002}
\todo{CS4002}
\todo[color=blue!40]{CS4003}
\todo[color=green!40]{CS4004}
El reensamblar objetos sólidos de 3 dimensiones tiene muchas aplicaciones en distintas disciplinas humanas, como por ejemplo en la arqueología. Reensamblar las diversas partes supone un trabajo delicado y tedioso, sin mencionar que si hablamos de objetos arqueológicos de gran valor su tiempo de manipulación es corto ello para no afectar su composición y por consiguiente su puesta en valor.\\

El reensamblaje implica conocer cuáles son las partes que guardan correspondencia y cómo debería verse el objeto final una vez ensamblado (objeto destino). Un problema común es no tener información acerca del objeto destino lo cual complica aún más la labor de correspondencia.\\

Diversos trabajos intentan resolver la correspondencia de las partes, algunos trabajan en base a la nube de puntos de 3 dimensiones de las partes \cite{chen2022neural} y otros a partir de una imagen \cite{li2020learning}. Si bien los anteriores trabajos muestran resultados prometedores varios tienen ciertas limitaciones, como por ejemplo la correspondencia solo se realiza entre 2 partes o dependen mucho del objeto final. Otros trabajos no dependen del objeto final, sino de toda la información semántica asociada a las partes.\\

En esta tesis buscamos realizar la correspondencia entre las distintas partes sólo a través de una nube de puntos en 3 dimensiones correspondiente a cada una de las partes.
 \\
 \newpage