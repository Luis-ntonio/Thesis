\subsection*{Large Language Models (LLMs)}
LLMs are significant advancements in natural language processing (NLP), evolving from statistical language models to neural models. The term "large language model" typically refers to pre-trained language models of considerable size, often containing hundreds of millions to billions of parameters \cite{zhao2023survey}.

These models are trained on massive text corpora using self-supervised learning methods, allowing them to generate human-like text, perform translation, summarization, sentiment analysis, and more. The extensive training data and advanced architectures enable LLMs to capture complex language patterns and exhibit remarkable zero-shot and few-shot learning abilities \cite{naveed2024comprehensive}.

LLMs are used in various fields beyond standard NLP tasks. They have shown promise in enhancing recommendation systems, performing complex planning, and even contributing to fields like telecommunications and robotics \cite{10305960} \cite{fan2023fatellm}.

\subsection*{Fine tuning}
Fine-tuning in large language models (LLMs) is a critical process that involves adjusting a pre-trained model's parameters to optimize performance on specific downstream tasks. This technique leverages the extensive training already performed on massive, unlabeled text corpora and then refines the model using a smaller, task-specific dataset. Fine-tuning is essential because it allows models to generalize from the vast, generic data they were initially trained on to the specialized tasks they need to perform. For instance, the Child-Tuning technique updates only a subset of parameters by masking out non-essential gradients, thus improving both efficiency and performance on tasks within the GLUE benchmark \cite{xu-etal-2021-raise}.

Fine-tuning strategies can vary significantly depending on the model and the resources available. For example, some approaches focus on parameter-efficient methods to reduce computational costs while maintaining high performance. Techniques like the LoRA (Low-Rank Adaptation) allow substantial fine-tuning of LLMs with minimal additional parameters, making it feasible even with limited computational resources \cite{sun2023comparative}. Additionally, methods such as differentially private fine-tuning have been developed to protect sensitive data during the fine-tuning process, achieving a balance between model utility and data privacy \cite{yu2022differentially}.

\subsection*{JSON-Tuning}
JSON-Tuning is an innovative approach designed to enhance the performance and efficiency of Large Language Models (LLMs) by utilizing the structured data representation capabilities of JSON (JavaScript Object Notation). This method leverages JSON's hierarchical structure to optimize the input-output processes of LLMs, resulting in improved parameter tuning and model interpretability. JSON-Tuning allows for more precise control over training data, which in turn leads to more robust and contextually accurate predictions. This approach facilitates the efficient organization of data, making it easier to manage and utilize during the training and fine-tuning phases of LLM development \cite{zheng2024llamafactory}.

The advantages of JSON-Tuning are not limited to performance enhancements alone. This technique can significantly reduce the computational burden typically associated with traditional fine-tuning methods. By streamlining data processing steps and minimizing redundancy, JSON-Tuning makes it feasible to deploy LLMs in real-time applications where speed and accuracy are crucial \cite{zhu2024lift}. Furthermore, the structured nature of JSON supports seamless integration with existing data pipelines and APIs, thus simplifying workflows for data scientists and developers. This synergy between structured data representation and advanced model tuning offers a promising direction for future research and development in machine learning.